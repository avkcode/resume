\documentclass[11pt]{article}
\usepackage{geometry}
\geometry{a4paper, margin=1in}
\usepackage{enumitem}
\usepackage{titlesec}
\usepackage{hyperref}

% Enable UTF-8 support
\usepackage[utf8]{inputenc}
\usepackage[T2A]{fontenc}
\usepackage[russian,english]{babel}

% Custom section formatting
\titleformat{\section}{\Large\bfseries}{\thesection}{1em}{}
\titleformat{\subsection}{\large\bfseries}{\thesubsection}{1em}{}

% Disable numbering for sections
\setcounter{secnumdepth}{0}

\begin{document}

% Header - Name
\begin{center}
    \textbf{\Huge Антон Крылов} \\
    \vspace{0.2cm}
    \href{mailto:antonvkrylov@yandex.ru}{antonvkrylov@yandex.ru} | +89819174839  \href{https://t.me/akrylof}{@akrylof} | \href{https://github.com/avkcode}{GitHub}
\end{center}

\section    
- \textbf{CI/CD Tools:} Jenkins, GitLab, TeamCity, Bitbucket, Nexus
- \textbf{Programming Languages:} Python, Go, Bash, Groovy
- \textbf{Infrastructure:} Docker, Kubernetes, Istio, PostgreSQL, ArgoCD
- \textbf{Other:} PCIDSS Compliance, Microservices Architecture, Automation, Oracle Cloud, AWS, Infrastructure as Code, GitOps

\makebox[\textwidth]{\rule{8cm}{0.4pt}}

\subsection{ООО "Омниплан"}
\textbf{Генеральный директор / Самозанятый} \hfill 11.2024 -- н.в. \\

Разработка и внедрение специализированных DevOps и DevSecOps решений для корпоративных клиентов:
- Создание безопасных CI/CD-конвейеров для финансового сектора
- Разработка инструментов автоматизации инфраструктуры
=- Внедрение практик "Infrastructure as Code"
- Консалтинг в области облачной безопасности
- Разработка кастомных решений для мониторинга и оркестрации

\subsection{ПАО "Сбербанк России"}
\textbf{Главный инженер по разработке} \hfill 06.2024 -- 11.2024 \\
Группа разработки управления развития технологий сопровождения бизнес-блоков Департамента ИТ Operations.

Разработал защищенный CI/CD-конвейер для миграции кредитного конвейера на микросервисную архитектуру, обеспечивающий:
- Изолированные окружения сборки с криптографической защитой
- Автоматическое сканирование уязвимостей (SAST/DAST)
- Цифровую подпись артефактов
- Соответствие требованиям PCI DSS и внутренним стандартам безопасности

Сократил время вывода обновлений в production с 2 недель до 1 дня. Покинул компанию для запуска собственного бизнеса в сфере разработки программного обеспечения.

\subsection{АО "Байкал Электроникс"}
\textbf{DevOps инженер} \hfill 02.2024 -- 04.2024 \\

Разрабатывал систему автоматизации для проектов российских процессоров и видеокарт на базе GitLab CI/CD. Создал пайплайны сборки и тестирования RTL-кода, интеграцию с EDA-инструментами и систему управления артефактами.

Покинул компанию после выполнения задач из-за разногласий с руководством по методологии работы.

\subsection{ООО "ГК "Иннотех"}
\textbf{Главный инженер DevOps} \hfill 29.04.2022 -- 03.02.2023 \\
Дивизион технологического развития платформенных решений (до 13.01.2023), затем Дивизион технологического развития управления рисками и ALM.

Разработал и внедрил единый CI/CD пайплайн для банка ВТБ, охватывающий более 150 микросервисов. Реализовал многоступенчатую систему сборки и деплоя с интеграцией в TeamCity, обеспечивающую безопасный и контролируемый процесс поставки ПО в production-окружение.

Создал систему автоматизации тестирования и валидации для платформы больших данных, включая интеграцию с Apache Spark и Hadoop-кластерами. Оптимизировал процессы обработки данных через реализацию GitOps-подхода с использованием Bitbucket Pipelines и ArgoCD.

\subsection{Oracle}
\textbf{Старший инженер-программист} \hfill 17.02.2020 -- 22.02.2022 \\

Разрабатывал системы автоматизации сборки Docker-образов на базе Oracle Linux, включая создание инструментов для генерации оптимизированных образов под enterprise-решения. Основной фокус - разработка системы управления жизненным циклом контейнеров в экосистеме Oracle.

Занимался адаптацией Oracle Linux под различные cloud-сценарии, включая настройку OL-образов для Oracle Cloud Infrastructure (OCI) и интеграцию с платформенными сервисами. Особое внимание уделял вопросам безопасности контейнеров и совместимости с Oracle-стеком технологий.

В рамках проектов по Oracle Linux участвовал в создании специализированных сборок для финансового сектора, обеспечивающих сертифицированную работу с базами данных Oracle в контейнеризованной среде. Разрабатывал инструменты для автоматического тестирования и валидации образов перед выкладкой в production.

Отдельное направление работы - миграция legacy-приложений на контейнерную платформу с сохранением совместимости с Oracle-специфичными функциями. Реализовывал решения по мониторингу и оркестрации контейнерных workloads в гетерогенных средах.

\subsection{Северо-Западный филиал ПАО "МегаФон"}
\textbf{Старший инженер} \hfill 23.09.2019 -- 12.02.2020 \\
ИТ инфраструктура Эксплуатация инфраструктуры и сервисов Технические инновации и инфраструктура.

- Разрабатывал и поддерживал Kubernetes-кластеры для корпоративных сервисов
- Настраивал и администрировал JFrog Artifactory как единый реестр артефактов
- Автоматизировал процессы развертывания и обновления инфраструктурных компонентов
- Участвовал в миграции legacy-систем на контейнерную платформу
- Оптимизировал процессы сборки и доставки артефактов

\textit{Ушел из компании, приняв предложение от Oracle Development с улучшенными финансовыми условиями и перспективами профессионального роста.}

\section{Предыдущий опыт}
\begin{itemize}
    \item \textbf{ООО "Тетасофт/Интервим"} (11.2017–08.2019) \\
    \textit{Инженер технической поддержки (Veeam Support)}
    \begin{itemize}
        \item Обеспечивал премиальную техническую поддержку продуктов Veeam для международных клиентов \textit{(вся работа велась исключительно на английском языке)}
        \item Решал комплексные кейсы для корпоративных заказчиков с инфраструктурой виртуализации (VMware vSphere, Microsoft Hyper-V)
        \item Работал с топовыми международными компаниями из списка Fortune 500
        \item Специализировался на траблшутинге сложных инцидентов резервного копирования в распределённых средах
        \item Разрабатывал технические решения для клиентов с парком 1000+ виртуальных машин
        \item Участвовал в миграционных проектах крупного масштаба
        \item Составлял техническую документацию и руководства на английском языке
    \end{itemize}
    \item \textbf{ООО "Рэйдикс"} (08.2017–11.2017) \\
    \textit{Инженер технической поддержки (Customer Success)}
    \begin{itemize}
        \item Работал с программно-определяемыми системами хранения данных (SDS) компании Raidix
        \item Настраивал и оптимизировал сложные кластерные инфраструктуры на базе Proxmox VE
        \item Консультировал заказчиков по архитектурным решениям для систем хранения данных
        \item Разрабатывал сценарии автоматизации развертывания и мониторинга
        \item Участвовал в пилотных проектах внедрения решений для крупных клиентов
        
        \textit{Покинул компанию, приняв более выгодное предложение от ООО "Тетасофт" с повышением зарплаты и перспективами профессионального роста}
    \end{itemize}
    \item \textbf{ООО "Импульс"} (09.2013–09.2017) \\
    \textit{Системный инженер / Сетевой инженер / Программист}
    \begin{itemize}
        \item Полный цикл разработки и поддержки ИТ-инфраструктуры региональной телерадиокомпании с ЦОД, распределённой сетью филиалов и вещательным оборудованием Harris Broadcasting.
    
        \item \textbf{Сетевая инженерия:}
        \begin{itemize}
            \item Проектирование и настройка LAN/WAN сетей (Cisco Catalyst, MikroTik RouterOS)
            \item Конфигурация VLAN, VPN, QoS, маршрутизации
            \item Мониторинг сети через Zabbix с кастомными шаблонами
            \item Устранение сложных сетевых инцидентов
        \end{itemize}
    
        \item \textbf{Системное администрирование:}
        \begin{itemize}
            \item Развертывание и настройка Zabbix для мониторинга серверов
            \item Создание кастомных дашбордов и триггеров в Zabbix
            \item Виртуализация на базе Proxmox VE
            \item Администрирование Linux/Windows серверов
        \end{itemize}
    
        \item \textbf{Автоматизация:}
        \begin{itemize}
            \item Интеграция Zabbix с другими системами через API
            \item Написание скриптов (Bash/Python) для обработки данных мониторинга
            \item Автоматизация отчётов по метрикам Zabbix
        \end{itemize}
    \end{itemize}
    \item \textbf{АО "Централизованный Региональный Технический Сервис"} (08.2013–09.2013) \\
    \textit{Сервисный инженер}
    \begin{itemize}
        \item Обслуживание и ремонт банкоматов в региональных отделениях банков
        \item Диагностика и устранение неисправностей банковского оборудования
    \end{itemize}
    \begin{itemize}
        \item \textbf{Филиал ПАО "МТС"/ЗАО "Комстар-Регионы"} (01.2013–07.2013) \\
        \textit{Монтажник волоконно-оптических линий связи (ВОЛС)}
        \begin{itemize}
            \item Выполнял монтаж и сварку оптических волокон на телекоммуникационных объектах
            \item Производил измерения параметров ВОЛС с помощью рефлектометров и других измерительных приборов
            \item Участвовал в строительстве магистральных и внутризоновых линий связи
            \item Осуществлял прокладку кабеля в кабельной канализации и по опорам ЛЭП
            \item Выполнял монтаж кроссового оборудования и организацию точек доступа
        \end{itemize}
\end{itemize}
\end{itemize}

\end{document}
